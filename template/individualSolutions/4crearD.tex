\documentclass[a4paper]{article}

\usepackage[spanish]{babel} % Le indicamos a LaTeX que vamos a escribir en espa�ol.
\usepackage[latin1]{inputenc} % Permite utilizar tildes y e�es normalmente
%\usepackage{framed}
\input{Algo1Macros}% Macros especificas para especificar problemas en AyEDI

\newcommand{\comen}[2]{%
\begin{framed}
\noindent \textsf{#1:} #2
\end{framed}
}
% Aca solo vamos a poner el esqueleto del documento, pero no vamos a especificar nada.

\begin{document} % Todo lo que escribamos a partir de aca va a aparecer en el documento.

\section{Tipos}

\input{tipos/tipos}

\section{Drone}

\input{tipos/drone}

\begin{problema}{crearD}{id: \ent, pd:[Producto]}{Drone}
\asegura{\neg enVuelo(\res)}
\asegura{|vueloRealizado(\res)| == 0}
\asegura{bateria(d) == 100}
\asegura{id(\res) == id}
\asegura{mismos(productosDisponibles(\res), pd)}
\end{problema}

%\section{Funciones Auxiliares}

%\input{espec/auxiliares.tex}

\end{document} %Termin�!
