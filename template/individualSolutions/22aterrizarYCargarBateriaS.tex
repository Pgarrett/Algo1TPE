\documentclass[a4paper]{article}

\usepackage[spanish]{babel} % Le indicamos a LaTeX que vamos a escribir en espa�ol.
\usepackage[latin1]{inputenc} % Permite utilizar tildes y e�es normalmente
%\usepackage{framed}
\input{Algo1Macros}% Macros especificas para especificar problemas en AyEDI

\newcommand{\comen}[2]{%
\begin{framed}
\noindent \textsf{#1:} #2
\end{framed}
}
% Aca solo vamos a poner el esqueleto del documento, pero no vamos a especificar nada.

\begin{document} % Todo lo que escribamos a partir de aca va a aparecer en el documento.

\section{Tipos}

\input{tipos/tipos}

\section{Sistema}

\input{tipos/sistema}

\begin{problema}{aterrizarYCargarBateriaS}{s: Sistema, b: \ent}{}

\requiere[bateriaOk]{0 \leq b \leq 100}
\modifica{s}
\asegura[mismoCampo]{campo(s) == campo(pre(s))}

\asegura[mismoEstado]{(\forall {p} \selec {parcelasConCultivo(campo(s))}) \\ estadoDelCultivo(s, prm(p), sgd(p)) == estadoDelCultivo(pre(s), prm(p), sgd(p))}

\asegura[dronesIguales]{(\forall {d} \selec {enjambreDrones(pre(s))}) \\ bateria(d) \geq b \Rightarrow droneEnLista(enjambreDrones(s), d)}

\asegura[dronesAterrizados]{(\forall {d} \selec {enjambreDrones(pre(s))}) \\ bateria(d) < b \Rightarrow (\exists d' \selec enjambreDrones(s)) \\ id(d') == id(d) \land bateria(d') == 100 \land \neg enVuelo(d') \land (mismos(productosDisponibles(d'),  productosDisponibles(d)))}

\end{problema}

\noindent \aux{droneEnLista}{ds: [Drone], d: Drone}{Bool}{(\exists d' \selec ds) (id(d) == id(d')) \land (bateria(d) == bateria(d')) \land (enVuelo(d) == enVuelo(d')) \land (vueloRealizado(d) == vueloRealizado(d')) \land (posicionActual(d) == posicionActual(d')) \land (mismos(productosDisponibles(d),  productosDisponibles(d')))}

%\section{Funciones Auxiliares}

%\input{espec/auxiliares.tex}

\end{document} %Termin�!
