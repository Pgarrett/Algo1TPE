\documentclass[a4paper]{article}

\usepackage[spanish]{babel} % Le indicamos a LaTeX que vamos a escribir en espa�ol.
\usepackage[latin1]{inputenc} % Permite utilizar tildes y e�es normalmente
%\usepackage{framed}
\input{Algo1Macros}% Macros especificas para especificar problemas en AyEDI

\newcommand{\comen}[2]{%
\begin{framed}
\noindent \textsf{#1:} #2
\end{framed}
}
% Aca solo vamos a poner el esqueleto del documento, pero no vamos a especificar nada.

\begin{document} % Todo lo que escribamos a partir de aca va a aparecer en el documento.

\section{Tipos}

\input{tipos/tipos}

\section{Sistema}

\input{tipos/sistema}

\begin{problema}{listoParaCosecharS}{s: Sistema}{\bool}

%Por invariante del campo, al menos una parcela tiene cultivo asi que no se indefine%
\asegura{res == (|[p | p \selec parcelasConCultivo(s), estadoDelCultivo(p) == ListoParaCosechar]|/|parcelasConCultivo(s)| >= 0.9)}

\end{problema}

\noindent \aux{parcelasConCultivo}{s: Sistema}{[(\ent, \ent)]}{[(i,j) | i \selec [0..prm(dimensiones(campo(s)))), j \selec [0..sgd(dimensiones(campo(s)))), contenido(campo(s), prm(p'), sgd(p')) == Cultivo]}

%\section{Funciones Auxiliares}

%\input{espec/auxiliares.tex}

\end{document} %Termin�!
