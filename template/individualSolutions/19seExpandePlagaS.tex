\documentclass[a4paper]{article}

\usepackage[spanish]{babel} % Le indicamos a LaTeX que vamos a escribir en espa�ol.
\usepackage[latin1]{inputenc} % Permite utilizar tildes y e�es normalmente
%\usepackage{framed}
\input{Algo1Macros}% Macros especificas para especificar problemas en AyEDI

\newcommand{\comen}[2]{%
\begin{framed}
\noindent \textsf{#1:} #2
\end{framed}
}
% Aca solo vamos a poner el esqueleto del documento, pero no vamos a especificar nada.

\begin{document} % Todo lo que escribamos a partir de aca va a aparecer en el documento.

\section{Tipos}

\input{tipos/tipos}

\section{Sistema}

\input{tipos/sistema}

\begin{problema}{seExpandePlagaS}{s: Sistema}{}
\modifica{s}
\asegura[mismoCampo]{campo(s) == campo(pre(s))}
\asegura[mismosDrones]{mismosDrones(enjambreDrones(s), enjambreDrones(pre(s)))}
\asegura[estadoDelCultivo]{(\forall {p} \selec {parcelasConCultivo(campo(s))}) \\ (seContagiaPlaga(pre(s), p) \land estadoDelCultivo(s, prm(p), sgd(p)) == ConPlaga) \\ \lor (\neg seContagiaPlaga(pre(s), p) \land estadoDelCultivo(s, prm(p), sgd(p)) == estadoDelCultivo(pre(s), prm(p), sgd(p)))}
\end{problema}

\aux{seContagiaPlaga}{s: Sistema, p: (\ent, \ent)}{Bool}{(\exists p' \selec p:parcelasAdyacentesConCultivo(campo(s), p)) \\ estadoDelCultivo(s, prm(p'), sgd(p')) == ConPlaga}

%\section{Funciones Auxiliares}

%\input{espec/auxiliares.tex}

\end{document} %Termin�!
