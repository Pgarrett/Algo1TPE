\documentclass[a4paper]{article}

\usepackage[spanish]{babel} % Le indicamos a LaTeX que vamos a escribir en espa�ol.
\usepackage[latin1]{inputenc} % Permite utilizar tildes y e�es normalmente
%\usepackage{framed}
\input{Algo1Macros}% Macros especificas para especificar problemas en AyEDI

\newcommand{\comen}[2]{%
\begin{framed}
\noindent \textsf{#1:} #2
\end{framed}
}
% Aca solo vamos a poner el esqueleto del documento, pero no vamos a especificar nada.

\begin{document} % Todo lo que escribamos a partir de aca va a aparecer en el documento.

\section{Tipos}

\input{tipos/tipos}

\section{Sistema}

\input{tipos/sistema}

\begin{problema}{despegarS}{s: Sistema, d: Drone}{}
\requiere[droneEnSistema]{droneEnLista(enjambreDrones(s), d)}
\requiere[droneNoEnVuelo]{\neg enVuelo(d)}
\requiere[algoDeBateria]{bateria(d) > 0}
\requiere[parcelaLibre]{\neg ((\forall p \selec parcelasAdyacentesAlGranero(campo(s))) \\ (\exists d' \selec enjambreDrones(s)) posicionActual(d') == p)}

\modifica{s}

\asegura[mismoCampo]{campo(s) == campo(pre(s))}

\asegura[mismoEstado]{(\forall {p} \selec {parcelasConCultivo(campo(s))}) \\ estadoDelCultivo(s, prm(p), sgd(p)) == estadoDelCultivo(pre(s), prm(p), sgd(p))}

\asegura[mismaCantidadDeDrones]{|enjambreDrones(s)| == |enjambreDrones(pre(s))|}

\asegura[noDmismosDrones]{(\forall d' \selec enjambreDrones(s), id(d') \neq id(d)) (\exists d'' \selec enjambreDrones(pre(s))) \\ mismoDrone(d', d'')}

\asegura[nuevoDrone]{(\exists d' \selec enjambreDrones(s)) id(d') == id(d)
\land bateria(d') == bateria(d)-1 \land enVuelo(d') \\ \land |vueloRealizado(d')| == 2 \land vueloRealizado(d')_0 == posicionGranero(campo(s)) \\ \land  mismos(productosDisponibles(d'),  productosDisponibles(d))}
\end{problema}

%\section{Funciones Auxiliares}

%\input{espec/auxiliares.tex}

\end{document} %Termin�!
