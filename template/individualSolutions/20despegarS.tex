\documentclass[a4paper]{article}

\usepackage[spanish]{babel} % Le indicamos a LaTeX que vamos a escribir en espa�ol.
\usepackage[latin1]{inputenc} % Permite utilizar tildes y e�es normalmente
%\usepackage{framed}
\input{Algo1Macros}% Macros especificas para especificar problemas en AyEDI

\newcommand{\comen}[2]{%
\begin{framed}
\noindent \textsf{#1:} #2
\end{framed}
}
% Aca solo vamos a poner el esqueleto del documento, pero no vamos a especificar nada.

\begin{document} % Todo lo que escribamos a partir de aca va a aparecer en el documento.

\section{Tipos}

\input{tipos/tipos}

\section{Sistema}

\input{tipos/sistema}

\begin{problema}{despegarS}{s: Sistema, d: Drone}{}
%Asumimos que modifica el drone d y es un error de tipeo%
\requiere{(\exists d' \selec enjambreDrones(s)) mismoDrone(d, d')}
\requiere{\neg enVuelo(d)}
\requiere{bateria(d) > 0}
%Aclarar que está en el granero por invariante de s%
\modifica{d}
\asegura{id(d) == id(pre(d))}
\asegura{bateria(d) == bateria(pre(d))-1}
\asegura{enVuelo(d)}
\asegura{(mismos(productosDisponibles(d),  productosDisponibles(pre(d))))}
\end{problema}

\noindent \aux{mismoDrone}{d1: Drone, d2: Drone}{Bool}{(id(d1) == id(d2)) \land (bateria(d1) == bateria(d2)) \land (enVuelo(d1) == enVuelo(d2)) \land (vueloRealizado(d1) == vueloRealizado(d2)) \land (posicionActual(d1) == posicionActual(d2)) \land (mismos(productosDisponibles(d1),  productosDisponibles(d2)))}

%\section{Funciones Auxiliares}

%\input{espec/auxiliares.tex}

\end{document} %Termin�!
