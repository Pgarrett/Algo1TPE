\documentclass[a4paper]{article}

\usepackage[spanish]{babel} % Le indicamos a LaTeX que vamos a escribir en espa�ol.
\usepackage[latin1]{inputenc} % Permite utilizar tildes y e�es normalmente
%\usepackage{framed}
\input{Algo1Macros}% Macros especificas para especificar problemas en AyEDI

\newcommand{\comen}[2]{%
\begin{framed}
\noindent \textsf{#1:} #2
\end{framed}
}
% Aca solo vamos a poner el esqueleto del documento, pero no vamos a especificar nada.

\begin{document} % Todo lo que escribamos a partir de aca va a aparecer en el documento.

\section{Tipos}

\input{tipos/tipos}

\section{Drone}

\input{tipos/drone}

\begin{problema}{vueloEscaladeroD}{d: Drone}{\bool}
\asegura{ res == |vueloRealizado(d)| \geq 3 \land primerasTresSonEscalera(vuelosRealizado(d)) \land (\forall \selector {i}{\rangoca{0}{|v|-1}}), seMueveComoLosPrimeroTres(vueloRealizado(d), i)}
\end{problema}

\noindent \aux{primerasTresSonEscalera}{ps: [(\ent, \ent)]}{\bool}{|\prm{restarPos(ps(d)_{2}, ps(d)_{0})}| == 1 \land |\sgd{restarPos(ps(d)_{2}, ps(d)_{0})}| == 1}
\noindent \aux{restarPos}{a,b: (\ent, \ent)}{(\ent, \ent)}{ (\prm{a}-\prm{b}, \sgd{a}-\sgd{b}}
\noindent \aux{seMueveComoLosPrimeroTres}{ps: [(\ent, \ent)], i: \ent}{\bool}{restarPos(ps_{i+1}, ps_{i}) == restarPos(ps_{(i \bmod 2) + 1}, ps_{i \bmod 2})}

%\section{Funciones Auxiliares}

%\input{espec/auxiliares.tex}

\end{document} %Termin�!
