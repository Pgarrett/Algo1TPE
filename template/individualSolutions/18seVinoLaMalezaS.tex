\documentclass[a4paper]{article}

\usepackage[spanish]{babel} % Le indicamos a LaTeX que vamos a escribir en espa�ol.
\usepackage[latin1]{inputenc} % Permite utilizar tildes y e�es normalmente
%\usepackage{framed}
\input{Algo1Macros}% Macros especificas para especificar problemas en AyEDI

\newcommand{\comen}[2]{%
\begin{framed}
\noindent \textsf{#1:} #2
\end{framed}
}
% Aca solo vamos a poner el esqueleto del documento, pero no vamos a especificar nada.

\begin{document} % Todo lo que escribamos a partir de aca va a aparecer en el documento.

\section{Tipos}

\input{tipos/tipos}

\section{Sistema}

\input{tipos/sistema}

\begin{problema}{seVinoLaMalezaS}{s: Sistema, ps: [(\ent,\ent)]}{}
\requiere{(\forall p \selec ps)  enRango(dimensiones(campo(s)), prm(p), sgd(p)}
\requiere{(\forall p \selec ps) contenido(campo(s), prm(p), sgd(p)) == Cultivo}
\modifica{s}
\asegura{campo(pre(s)) == campo(s)}
\asegura{mismos(enjambreDrones(pre(s)), enjambreDrones(s))}
\asegura{(\forall p \selec ps) estadoDelCultivo(s, prm(p), sgd(p)) == ConMaleza}
\end{problema}

%\section{Funciones Auxiliares}

%\input{espec/auxiliares.tex}

\end{document} %Termin�!
