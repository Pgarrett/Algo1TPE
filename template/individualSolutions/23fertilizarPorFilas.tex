\documentclass[a4paper]{article}

\usepackage[spanish]{babel} % Le indicamos a LaTeX que vamos a escribir en espa�ol.
\usepackage[latin1]{inputenc} % Permite utilizar tildes y e�es normalmente
%\usepackage{framed}
\input{Algo1Macros}% Macros especificas para especificar problemas en AyEDI

\newcommand{\comen}[2]{%
\begin{framed}
\noindent \textsf{#1:} #2
\end{framed}
}
% Aca solo vamos a poner el esqueleto del documento, pero no vamos a especificar nada.

\begin{document} % Todo lo que escribamos a partir de aca va a aparecer en el documento.

\section{Tipos}

\input{tipos/tipos}

\section{Sistema}

\input{tipos/sistema}

\begin{problema}{fertilizarPorFilas}{s: Sistema}{}
\requiere[unDronVolandoPorFila]{sinRepetidos([sgd(posicionActual(d)) | d \selec enjambreDrones(s), enVuelo(d)])}

\modifica{s}

\asegura[mismoCampo]{campo(s) == campo(pre(s))}

\asegura[parcelasSinFertilizar]{(\forall p \selec parcelasConCultivo(campo(pre(s))), p \notin parcelasAFertilizar(pre(s))) \\ estadoDelCultivo(s, prm(p), sgd(p)) == estadoDelCultivo(pre(s), prm(p), sgd(p))}

\asegura[parcelasFertilizadas]{(\forall p \selec parcelasAFertilizar(pre(s))) \\ estadoDelCultivo(pre(s), prm(p), sgd(p)) \in [EnCrecimiento, RecienSembrado] \\ \Rightarrow estadoDelCultivo(s, prm(p), sgd(p)) == ListoParaCosechar) \\
estadoDelCultivo(pre(s), prm(p), sgd(p)) \notin [EnCrecimiento, RecienSembrado] \\ \Rightarrow estadoDelCultivo(s, prm(p), sgd(p)) == estadoDelCultivo(pre(s), prm(p), sgd(p))}

\asegura[mismaCantidadDeDrones]{|enjambreDrones(s)| == |enjambreDrones(pre(s))|}

\asegura[siNoVuelanNoSeModifican]{(\forall d \selec enjambreDrones(pre(s)), \neg enVuelo(d)) \\ droneEnLista(enjambreDrones(s), d)}

\asegura[siVuelanFertilizan]{(\forall d \selec dronesEnVuelo(pre(s))) (\exists {d'} \selec {enjambreDrones(s)}) \\ 
id(d') == id(d) \land bateria(d') == bateria(d)-|parcelasAFertilizarPorDrone(s, d)|
\land \neg enVuelo(d') 
\\
\land mismos(productosDisponibles(d'), eliminarFertilizante(d, |parcelasAFertilizarPorDrone(s, d)|))
}

\end{problema}

\aux{parcelasAFertilizar}{s: Sistema}{[(\ent, \ent)]}{[p | d \selec enjambreDrones(s), \\ p \selec parcelasAFertilizarPorDrone(s, d), enVuelo(d)]}

\aux{parcelasAFertilizarPorDrone}{s: Sistema, d: Drone}{[(\ent, \ent)]}{[(i, sgd(posicionActual(d))) | \\ i \selec [fertilizaHastaColumna(s, d)..prm(posicionActual(d)))]}

\aux{fertilizaHastaColumna}{s: Sistema, d: Drone}{\ent}{max0([primerObstaculoEnColumna(campo(s), d)+1, \\ prm(posicionActual(d))-bateria(d), prm(posicionActual(d))-fertilizanteDisponible(d)])}

\aux{primerObstaculoEnColumna}{c: Campo, d: Drone}{\ent}{max0([i | i \selec [0..prm(posicionActual(d))), \\ contenido(c, i, sgd(posicionActual(d))) \neq Cultivo])}

\aux{eliminarFertilizante}{d: Drone, cant: \ent}{[Producto]}{[p | p \selec ps, p \neq Fertilizante]\\++[Fertilizante | i \selec [0..|fertilizanteDisponible(d)|-cant)]}

%\section{Funciones Auxiliares}

%\input{espec/auxiliares.tex}

\end{document} %Termin�!
