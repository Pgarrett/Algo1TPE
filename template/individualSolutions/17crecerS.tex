\documentclass[a4paper]{article}

\usepackage[spanish]{babel} % Le indicamos a LaTeX que vamos a escribir en espa�ol.
\usepackage[latin1]{inputenc} % Permite utilizar tildes y e�es normalmente
%\usepackage{framed}
\input{Algo1Macros}% Macros especificas para especificar problemas en AyEDI

\newcommand{\comen}[2]{%
\begin{framed}
\noindent \textsf{#1:} #2
\end{framed}
}
% Aca solo vamos a poner el esqueleto del documento, pero no vamos a especificar nada.

\begin{document} % Todo lo que escribamos a partir de aca va a aparecer en el documento.

\section{Tipos}

\input{tipos/tipos}

\section{Sistema}

\input{tipos/sistema}

\begin{problema}{crecerS}{s: Sistema}{}
\modifica s
\asegura[mismoCampo]{campo(s) == campo(pre(s))}
\asegura[mismosDrones]{mismosDrones(enjambreDrones(s), enjambreDrones(pre(s)))}

\asegura{(\forall p \selec parcelasConCultivo(s))  deRecienSembradoAEnCrecimiento(pre(s), s, p) \\ \lor deEnCrecimientoAListoParaCosechar(pre(s), s, p) \lor mismoEstado(pre(s), s, p)}

\end{problema}

\aux{deRecienSembradoAEnCrecimiento}{preS, s:Sistema, p: (\ent, \ent)}{Bool}{(estadoDelCultivo(preS, prm(p), sgd(p)) \\ == RecienSembrado \land estadoDelCultivo(s, prm(p), sgd(p)) == EnCrecimiento)}

\aux{deEnCrecimientoAListoParaCosechar}{preS, s:Sistema, p: (\ent, \ent)}{Bool}{(estadoDelCultivo(preS, prm(p), sgd(p)) \\ == EnCrecimiento \land estadoDelCultivo(s, prm(p), sgd(p)) == ListoParaCosechar)}

\aux{mismoEstado}{preS, s:Sistema, p: (\ent, \ent)}{Bool}{(estadoDelCultivo(preS, prm(p), sgd(p)) \\ \notin [RecienSembrado, EnCrecimiento] \land estadoDelCultivo(s, prm(p), sgd(p)) == estadoDelCultivo(preS, prm(p), sgd(p)))}

%\section{Funciones Auxiliares}

%\input{espec/auxiliares.tex}

\end{document} %Termin�!
