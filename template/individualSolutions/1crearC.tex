\documentclass[a4paper]{article}

\usepackage[spanish]{babel} % Le indicamos a LaTeX que vamos a escribir en espa�ol.
\usepackage[latin1]{inputenc} % Permite utilizar tildes y e�es normalmente
%\usepackage{framed}
\input{Algo1Macros}% Macros especificas para especificar problemas en AyEDI

\newcommand{\comen}[2]{%
\begin{framed}
\noindent \textsf{#1:} #2
\end{framed}
}
% Aca solo vamos a poner el esqueleto del documento, pero no vamos a especificar nada.

\begin{document} % Todo lo que escribamos a partir de aca va a aparecer en el documento.

\section{Tipos}

\input{tipos/tipos}

\section{Campo}

\begin{problema}{crearC}{posG, posC: (\ent, \ent)}{Campo}
\end{problema}

\begin{problema}{dimensionesC}{c: Campo}{(Ancho, Largo))}
\end{problema}

\begin{problema}{contenidoC}{c: Campo, i, j: \ent}{Parcela}
\end{problema}


\begin{problema}{crearC}{posG, posC: (\ent, \ent)}{Campo}
\requiere{ posPositiva(posG) \land posPositiva(posC) }
\requiere{ \prm{posG} \neq \prm{posC} \lor \sgd{posG} \neq \sgd{posC}}
\requiere{ distancia((0,0), posG) \leq 100 \land distancia(posG, posC) \leq 100}

\asegura{ contenido(res, \prm{posG}, \sgd{posG}) == Granero} %hace falta 'importar' un tipo?
\asegura{ contenido(res, \prm{posC}, \sgd{posC}) == Casa} 
\end{problema}

\noindent \aux{posPositiva}{pos: (\ent, \ent)}{\bool}{\prm{pos} \geq 0 \land \sgd{pos} \geq 0}
\noindent \aux{distancia}{a, b: (\ent, \ent)}{\ent}{|\prm{a} - \prm{b}| + |\sgd{a} - \sgd{b}|} %se puede usar la del enunciado?

%\section{Funciones Auxiliares}

%\input{espec/auxiliares.tex}

\end{document} %Termin�!
