\documentclass[a4paper]{article}

\usepackage[spanish]{babel} % Le indicamos a LaTeX que vamos a escribir en espa�ol.
\usepackage[latin1]{inputenc} % Permite utilizar tildes y e�es normalmente
%\usepackage{framed}
\input{Algo1Macros}% Macros especificas para especificar problemas en AyEDI

\newcommand{\comen}[2]{%
\begin{framed}
\noindent \textsf{#1:} #2
\end{framed}
}
% Aca solo vamos a poner el esqueleto del documento, pero no vamos a especificar nada.

\begin{document} % Todo lo que escribamos a partir de aca va a aparecer en el documento.

\section{Tipos}

\input{tipos/tipos}

\section{Drone}

\input{tipos/drone}


\begin{problema}{vuelosCruzadosD}{ds: [Drone]}{[((\ent,\ent),\ent)]}
\requiere{(\forall \selector {d}{ds}) enVuelo(d)}
\requiere{(\forall \selector {d1}{ds}, \selector {d2}{ds}) |vueloReaizado(d1)| == |vueloRealizado(d2)|}
\asegura{TO-DO}
\asegura{ordenadoAsc(res) \lor ordenadoDesc(res)}
\end{problema}

\noindent \aux{dronesEnPosicionEnInstante}{ds: [Drone], pos: (\ent, \ent), i: \ent}{\ent}{|[d | d \selec ds, vueloRealizado(d)_i == pos]|}
\noindent \aux{ordenadoAsc}{[((\ent,\ent),\ent)]}{Bool}{(\forall \selector {p}{[0..|res|-1)}) sgd(res_p) \leq sgd(res_{p+1}))}
\noindent \aux{ordenadoDesc}{[((\ent,\ent),\ent)]}{Bool}{(\forall \selector {p}{[0..|res|-1)}) sgd(res_p) \geq sgd(res_{p+1}))}

%\section{Funciones Auxiliares}

%\input{espec/auxiliares.tex}

\end{document} %Termin�!
